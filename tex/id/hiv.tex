\documentclass[main.tex]{subfiles}
\begin{document}

\section{Diagnosis}


\subsection{Initial Testing}

\begin{itemize}[noitemsep]
	\item HIV Ig Testing
	\item CD4 Count
	\item Viral Load (Circulating HIV RNA)
	\item Hepatitis Testing
	\item Genotypic resitance testing
	\item Basic Labs
\end{itemize}

\subsubsection{Drug Resistance Testing}
\begin{itemize}[noitemsep]
	\item Therapy Naive Patients
		\item Entry to care
		\item Geontypic preferred to phenotypic
		\item Typically RT and PR genes, not INSTI genes unless concern for INSTI resistance
	\item Therapy Experienced Patients
		\item Virologic failure and RNA > 1000 \si{copies/mL}
		\item RNA (500,1000), testing may be unsuccessfuly but should be considered
		\item Pts with suboptimal viral load reduction
		\item Failure of INSTI regimen
		\item Geontypic preferred to phenotypic
			\item Phenotypic can be added if complex resistance patterns expected
\end{itemize}


\todo[inline]{Consider adding Table 3: Timepoint or Frequency of Testing}

\subsection{Follow-Up Testing}

\begin{itemize}[noitemsep]
	\item Viral load in 2-4 wks (no later than 8 wks) post intervention to determine viral response
	\item Viral load 4-8 wks after regimen changes
	\item Viral load Q3-4mo on stable regimens
	\item CD4 Q3-6mo
\end{itemize}


\section{Drug Classes}

\begin{table}
	\centering
	\caption*{}
	\begin{tabular}{l l l}
		\textbf{Generic} & \textbf{Brand} & \textbf{Abbreviation} \\ \hline
		\multicolumn{3}{c}{NRTI} \\
		
	\end{tabular}

\end{table}

\section{Drug Therapy}

\subsection{Goals of Therapy}
\begin{itemize}[noitemsep]
	\item Prevent OIs
	\item Maintain virologic suppression (\textless 200 \si{copies/mL} $\implies$ not transmissible) 
\end{itemize}

\subsection{When to delay therapy}
\begin{itemize}[noitemsep]
	\item Concern for IRIS (immune reconstitution inflammatory syndrome) in pts with cryptococcal or TB meningitis
\end{itemize}

\subsection{Testing Prior to Drug Therapy}
\begin{itemize}[noitemsep]
	\item Pregnancy test for all female pts of childbearing age 
	\item HLA B*5701 if considering ABC-containing regimens
\end{itemize}

\subsection{Guideline-Recommended Initial Therapies}
\begin{itemize}[noitemsep]
	\item Bictegravir / TAF / Emtracitabine
	\item Dolutegravir / abacavir / lamivudine
		\item HLA-B*5701 (-) and no HBV
	\item Dolutegravir + (Emtricitabine or Lamivudine) + (TAF or TDF)
	\item Dolutegravir/Lamivudine
		\item CI w/ HBV, RNA > 500k \si{copies/mL}, initiation before results of RT resistance testing are available
	\item Raltegravir + (Emtracitabine or Lamivudine) + (TAF or TDF)
		\item RAL containing regimens have lower barrier to resistance than others
\end{itemize}

\subsection{Other Treatment Considerations}
\begin{itemize}[noitemsep]
	\item INSTI regimens are the best tolerated
	\item PI regimens should have resistance testing conducted if possible, and DRV has the lowest rates of resistance
	\item NNRTIs have low barrier to resistance
	\item Boosted PI regimens have the most interactions
\end{itemize}

\begin{table}[h]
	\centering
	\rowcolors{2}{white}{tableColor}
	\caption{Major Therapy Considerations}
	\begin{tabular}{c c c c}
		\textbf{Drug} & \textbf{Pill Burden} & \textbf{Renal Fxn} &  \textbf{Food} & \textbf{Major SEs} \\ \hline
		Biktarvy & 1 QD & & & \\ 
		Stribild & 1 QD & \textgreater 70 & & \\
		Genvoya & 1 QD & \textgreater 30 or HD  & &  \\

	\end{tabular}
\end{table}

\begin{table}[h]
	\centering
	\rowcolors{2}{white}{tableColor}
	\caption{Major Drug Interactions}
	\begin{tabular}{c m{4in}}
		\textbf{Agent} & \textbf{Interactions} \\ \hline
	\end{tabular}
\end{table}

\nocite{centersfordiseasecontrolandpreventionGuidelinesPreventionTreatment2019,centersfordiseasecontrolandpreventionGuidelinesUseAntiretroviral2019}

\printmybib
\end{document}
